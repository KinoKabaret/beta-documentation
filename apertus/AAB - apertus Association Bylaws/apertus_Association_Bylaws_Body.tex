
\section{Name, Location and Field of Action}

The full Association name is "Apertus Association - supporting open and free audiovisual media and technology/Verein zur Förderung offener und freier audiovisueller Medien und Technologie."\\

\noindent The Association operates worldwide and is registered and headquartered in Vienna, Austria.\\




\section{Purpose}

The Association is a non-profit organisation and exclusively serves the public good as defined by sections 34 to 47 of the Austrian Federal Fiscal Code.\\

The aim of the Association is to support the creation and distribution of knowledge, and to establish networks around open source, audiovisual media production technologies.\\

All Association project undertakings are strictly made free (in terms of liberty) around open source software and open hardware licenses with any and all related knowledge being made publicly available.\\




\section{Means to Achieve Association Purpose}

The Association’s purpose shall be achieved through the following methods:

\begin{enumerate}
\item Creating and operating communication platforms that facilitate cooperation and collaboration with 
other national and international (intercultural exchange) associations, groups and individuals motivated by similar goals.
\item Talks, gatherings, conferences, workshops and discussions.
\item Facilitating the creation of audiovisual productions. 
\item Exhibiting and distributing audiovisual content.
\item Supporting, developing and manufacturing products that might realise Association goals.
\item Supporting Association members and non-members, e.g. artists, researchers, developers, creatives, scientists and visionaries.
\item The publication of materials.
\end{enumerate}

The above methods shall be funded in the following ways:

\begin{enumerate}
\item Donations, financial contributions from members and non-members, legacies, sponsoring, and membership fees where applicable.
\item Grants, prize money, and state or privately owned art/cultural institution funding.
\item Income and or reimbursement of costs levied through talks, gatherings, counseling, events and their respective projects.
\item The sale of any audiovisual content adhering to Association goals.
\item Leasing space and equipment in relation to any endeavors adhering to Association goals.
\item The sale of products adhering to Association goals.
\item Income accrued through the selling of shares held in third party corporate entities or Association owned ventures in accordance with and subject to relevant laws.
\end{enumerate}

The Association must ensure that fundraising is motivated by the benefit of the wider public. Competing with other businesses of a similar nature must only be undertaken if the Association’s failure to do so might jeopardise the fulfillment of the Association’s purposes. Surpluses from all activities must exclusively and directly serve the charitable purposes of the Association. Members of the Association may not profit from Association funding methods (this also applies when members leave the Association voluntarily or are dismissed by the Association). No person, member or otherwise, may benefit from disproportionately high remuneration of expenditures through the Association.\\





\section{Types of Membership}


\begin{enumerate}
\item The Association consists of three member types:

\begin{enumerate}
\item Ordinary Members.
\item Supporting Members.
\item Honorary Members.

\end{enumerate}
\item To be considered for Ordinary Membership non-members must demonstrate that they are willing to advance and achieve Association goals. Ordinary Members are entitled to vote in association polls.
\item Supporting Members support the Association financially through the paying of membership fees and/or by endorsing the Association and its goals/purposes. Supporting Members are not entitled to vote in Association polls.
\item Honorary Memberships are awarded to people who have contributed to the fulfilment of Association goals in a major way.\\ Honorary Members are chosen by the Board of Directors and do not get to vote in Association polls.

\end{enumerate}

Membership types are non exclusive, i.e. a single person may hold more than one membership-type simultaneously.\\





\section{Obtaining Membership Status}

Any person or legal entity, regardless of their geographic location, may apply for Ordinary and/or Supporting Membership in written form (electronic data processing methods are considered valid).\\

The Association Board of Directors will determine whether or not applications are successful. At least half of the Board of Directors are required in order to reach a decision with a 2/3 majority determining the outcome. Board of Director decisions, which must be forwarded to the applicant immediately, are not required to detail the reasoning behind any decision making.\\




\section{Ending Membership}

A person’s Ordinary, Supporting, or Honorary Membership status would be concluded in the event of their death, their decision to leave voluntarily by giving notice in written form (electronic data processing methods are considered valid), or in the event of a decision made by the Association Board of Directors to terminate hitherto held membership status.\\

Ordinary members have the right to request that the Association Board of Directors consider terminating another Ordinary, Honorary or Supporting Member’s membership status.\\




\section{Rights and Responsibilities of Members}

\begin{enumerate}
\item Every member has the right to request the current bylaws from the Association Board of Directors.
\item Any member holding an Ordinary Membership possesses an active and passive voting right in polls and General Assemblies.
\item Any Ordinary Member has the right to call for an Extraordinary General Assembly. Whether or not an an Extraordinary General Assembly is held requires the support of 25\% of Ordinary Members.
\item In a General Assembly all members are to be informed of the activity and financial situation of the Association by the Association Board of Directors.
\item All members are to be informed of the completion of a statement of accounts by the Association Board of Directors.
\item All members are obliged to follow Association bylaws, uphold any decisions made by the Association Board of Directors, promote what the Association stands for, and refrain from doing anything that might harm Association managed project development or the public perception thereof.
\item Every Ordinary Member is responsible for informing the association of his/her current email address (and any changes thereof).
\item Supporting and Honorary Members do not posses an active and passive voting right.
\end{enumerate}




\section{Institutions of the Association}

Institutions refer to the General Assembly (Sections 9 and 10), Board of Directors (Sections 11 to 13), Financial Auditor (Section 14) and an Arbitral Tribunal (Section 15)\\




\section{General Assembly}


\begin{enumerate}
\item The General Assembly is a meeting of all members as outlined in the Austrian “Association Law of 2002”. An ordinary General Assembly happens at least once every four years.
\item An Extraordinary General Assembly takes place within a period of no more than four weeks from:

\begin{enumerate}
\item A ruling made by the Association Board of Directors or a General Assembly of Ordinary Members.
\item A call for a General Assembly, in written form, made by at least 25\% of Ordinary Members (electronic data processing methods are considered valid).
\item A request from the Accountant (Austrian Law: Section 21 Abs. 5 erster Satz Vereinsgesetz).
\item A request from the Financial Auditor (Section 11.2 fouth sentence of these bylaws, Austrian law: Section 21 Abs. 5 zweiter Satz Vereinsgesetz).
\item A request from a Court Ordered Trustee (Austrian Law: Section 21 Abs. 5 zweiter Satz Vereinsgesetz).
\end{enumerate}

\item All members are to be notified in written form (electronic data processing methods are considered valid) at least five days in advance of Ordinary or Extraordinary General Assemblies. The call for a General Assembly must contain a respective agenda.
\item Any Ordinary Member can request, in written form (electronic data processing methods are considered valid) addressed to Association Board of Directors and at least two days in advance of a General Assembly, additions to be made to the agenda pertaining to a respective General Assembly.
\item During General Assemblies, other than the call for an Extraordinary General Assembly, valid rulings can only be made over the items listed on a General Assembly’s respective agenda.
\item At a General Assembly all members have a right to participate and all Ordinary Members are considered eligible for a right to vote. Each member has one vote. Transferring the right to vote to another member via a written delegation of authority, electronic or otherwise, is not permitted.
\item The General Assembly is to be held in a purely electronic way (Internet) and participation must not require registration across paid services (excluding an ISP which is accepted as being required to connect to the Internet).
\item For General Assembly rulings to be considered valid a minimum of 25\% of Ordinary Members must be present.
\item In general all polls in the General Assembly require an absolute majority of valid votes. Polls to change the Association bylaws or to decide upon the voluntary dissolving of the Association require a 2/3 majority of valid votes.
\item The Chairman or Vice Chairman holds and mediates over the general assembly.
 \ldots
\end{enumerate}




\section{Functions of the General Assembly}

The function of the General Assembly is as follows:

\begin{enumerate}
\item Discussing and/or making decisions over items listed on the respective agenda.
\item Reading and approval of the statement of accounts with the Association’s Financial Auditor/s.
\item Voting on the dismissal of Accountants or individuals hitherto classed as Directors.
\item Approving legal and financial auditing matters concerning the Association.
\item Exoneration of the Board of Directors.
\item Determining joining and membership fees where applicable.
\item Creditation or derecognition of honorary memberships.
\item Approving changes to Association bylaws.
\item Administrating the voluntary dissolving of the Association.
\end{enumerate}





\section{Board of Directors}


\begin{enumerate}
\item The Board of Directors must consist of at least two members: Chairman and Vice Chairman. The Board of Directors should also consist of an Advisory Board.
\item The Advisory Board can consist of up to ten ordinary members.
\item The Board of Directors is elected by a General Assembly. If a member of the Board leaves the Board of Directors then he/she has the right to select another member to take his/her place immediately, but the next General Assembly has to approve the new Board member. If the leaving member does not select a successor then the Chairman/Vice Chairman can select another member to take his/her place immediately but the next General Assembly has to approve the decision. If the Board of Directors is reduced to less than two people then the Financial Auditor has to call for an Extraordinary General Assembly so that a new Board of Directors can be appointed immediately. If the Financial Auditor is unable to act then any Ordinary Member is eligible to call for a Court Ordered Trustee to hold an Extraordinary General Assembly so that a new Board of Directors can be elected.
\item The term of office of the Board of Directors is four years; reelection is possible. Each role in the Board of Directors is to be executed personally.
\item The Board of Directors is inaugurated by the Chairman or in his/her absence the Vice Chairman in written form (electronic data processing methods are considered valid).
\item The Board of Directors is able to make decisions whereupon all Directors have been invited and at least 2/3 of Board members participate in the decision making process (vote).
\item The Board of Directors makes decisions with absolute majority. In the event of a tie the Chairman has the final say.
\item The Board of Directors is lead by the Chairman/Vice Chairman.
\item The function of each Board Member ends upon: death, end of term, resignation, dismissal.
\item The General Assembly can at any time dismiss individual Board members or the entire Board of Directors. The dismissal takes effect from such time that a new Board Member or an entire Board are elected.
\item Any Association Board of Directors member can announce his resignation in written form to the Board of Directors at any time (electronic data processing methods are considered valid). If the entire Board of Directors announce their resignation then the General Assembly must be notified. In the case of a single Board member’s resignation the resignation takes effect immediately upon a successor being either elected by the leaving member or by the Chairman, otherwise the resignation takes effect from such time that a new Board Member or an entire Board are elected.
\end{enumerate}




\section{Functions of the Board of Directors}

The Association Board of Directors run the Association (Austrian “Vereinsgesetzes 2002”). All functions not covered in the bylaws or the functions of other Association institutions are the responsibility of the Board of Directors. These are as follows:

\begin{enumerate}
\item Establishing accounting standards which match the Association's activities with constant documentation/monitoring of income and expenses.
\item Keeping a current inventory of Association held assets in accordance with local law.
\item Creation of an Association annual expenses estimate and statement of accounts.
\item Preparation and call for General Assembly in cases outlined in Section 9.2.
\item Keeping Association members informed of Association activity, performance and the forwarding of an approved statement of accounts to the aforementioned.
\item Managing Association funds.
\item Acceptance and dismissal of Ordinary, Supporting and Honorary Association members.
\item The hiring and firing of Association employees.
\end{enumerate}

     


\section{Special Tasks of Certain Board of Director Positions}

\begin{enumerate}
\item The Chairman leads the business of the Association.
\item The Chairman conveys Association rulings to the public. Written statements require the signature of the Chairman. A General Assembly can elect a member of the Board of Directors as financial representative.
\item Acts of law between a Board member and the Association require the approval of 2/3 of the other Board members. Certificates of authority for acts of law (like signing written statements or conveying Association rulings to the public) for other members of the Board can be approved/signed by the Chairman.
\item In case of imminent danger the Chairman is entitled to execute functions normally held by other board members at his/her own discretion, but must report to and get his/her actions approved by the respective Board members.
\item The Chairman leads the General Assembly and Association Board of Directors.
\item In case of absence the Vice Chairman assumes the role of Chairman.
\item The Advisory Board guides the direction/strategic development of the Association. Advisory Board members must hold ordinary membership status for more than one year before being eligible for Board membership. Advisory Board members take part in Board of Director votes and can also assume the roles of other Board of Directors when required. Advisory Board members are elected in a different process to that of other Board of Director positions, i.e. Individual elections for each candidate. In this instance a candidate who receives more than 50\% approval votes is elected. If there are more candidates than seats then the highest approval rate determines who gets elected. In case of draws a runoff determines the winner.
\end{enumerate}




\section{Financial Auditor}

Two Financial Auditors are to be elected by the General Assembly for each term of four years (reelection is possible). The Financial Auditors are not allowed to hold any other functions in the Association other than their participating in General Assembly.\\

The Financial Auditor's responsibilty is to periodically review the Association’s businesses, finances, funds and any assets held - especially, as defined in these bylaws, in terms of spending in relation to Association methods and goals. The Board of Directors must supply Financial Auditors with any and all required documentation. Financial Auditors must report the results of their review to the Association Board of Directors.\\

Acts of law between Financial Auditors and the Association require the approval of a General Assembly.\\




\section{Arbitral Tribunal}

For settlement of all Association related disputes an Association Internal Arbitral Tribunal is scheduled.\\     

The Arbitral Tribunal is formed by three Ordinary Association Members. The Board of Directors asks one disputing group to elect the first member of the Arbitral Tribunal within five days. Once the decision is made the Board of Directors ask the second disputing group to elect their member of the Arbitral Tribunal within ten days. After the Board has been informed which members have been elected the two Arbitral Tribunal members select a third Ordinary Member as mediator within ten days. If no single mediator is named then one of the two hitherto elected members are chosen at random to assume the role. The members of the Arbitral Tribunal must not hold any other position in the Association besides having participated in the General Assembly relating to the dispute.\\

The mediator listens to both groups’ claim/defense and then proposes a resolution in written form within ten days. Both disputing groups can accept or reject the proposal in the form of a written statement within ten days. This statement will be supplied to the mediator and must outline the reasons for this rejection. Upon rejection the mediator has ten days to propose a second resolution which is considered binding.\\

If the disputing parties go to ordinary court without seeking resolution through an Association Internal Arbitral Tribunal first then the court is expected to reject the claims and the suing party must pay any and all resulting expenses and fines. Disputing groups may not settle any claims in ordinary court before a period of six months after the Arbitral Tribunal has completed its work has elapsed.\\




\section{Voluntary Dismissal of the Association}

The voluntary dismissal of the Association can be achieved with a 2/3 majority of valid votes being counted in a General Assembly. In this case the General Assembly has to name a dismissal representative who is then put in charge of completing the dismissal of the Association. The General Assembly also has to decide what to do with any remaining (after paid liabilities) Association funds and assets. In the case of dismissal the remaining Association funds must not, whether directly or indirectly, be given to Association members and must be spent entirely on a charitable purpose as defined by Subsection 34ff BAO.\\

The last Board of Directors has to report the voluntary dismissal of the Association to the government authority (Austrian Vereinsbehörde) within a period of four weeks after the date that the decision to do so was made.


