\section{Converting}

Overview text required.




\subsection{RAW12 to PGM}

Converts a video file recorded in AXIOM raw to a PGM image sequence and applies the darkframe (needs to be created beforehand).\\

Currently clips must go through ffmpeg before hdmi4k can read them:\\ 

\consoleCommand{ffmpeg -i CLIP.MOV -c:v copy OUTPUT.MOV}

To cut out a video between IN and OUT with ffmpeg but maintaing the original encoding data:

\consoleCommand{ffmpeg -i CLIP.MOV -ss IN\_SECONDS -t DURATION\_SECONDS -c:v copy OUTPUT.MOV}


\begin{lstlisting}[language=bash,morekeywords=$,keywordstyle=\bfseries,frame=none,xleftmargin=.25in,belowskip=2em, aboveskip=2em]
	hdmi4k
    HDMI RAW converter for Axiom BETA
     
    Usage:
      ./hdmi4k clip.mov
      raw2dng frame*.pgm [options]
     
    Calibration files:
      hdmi-darkframe-A.ppm, hdmi-darkframe-B.ppm:
      averaged dark frames from the HDMI recorder (even/odd frames)
     
    Options:
    -                   : Output PGM to stdout (can be piped to raw2dng)
    --3x3               : Use 3x3 filters to recover detail (default 5x5)
    --skip              : Toggle skipping one frame (try if A/B autodetection fails)
    --swap              : Swap A and B frames inside a frame pair (encoding bug?)
    --onlyA             : Use data from A frames only (for bad takes)
    --onlyB             : Use data from B frames only (for bad takes)
\end{lstlisting}





\subsection{RAW12 to DNG}

Converts AXIOM Beta raw image to DNG. 

   
\begin{lstlisting}[language=bash,morekeywords=$,keywordstyle=\bfseries,frame=none,xleftmargin=.25in,belowskip=2em, aboveskip=2em]
DNG converter for Apertus .raw12 files
 
Usage:
  ./raw2dng input.raw12 [input2.raw12] [options]
  cat input.raw12 | ./raw2dng output.dng [options]
 
Flat field correction:
 - for each gain (N=1,2,3,4), you may use the following reference images:
 - darkframe-xN.pgm will be subtracted (data is x8 + 1024)
 - dcnuframe-xN.pgm will be multiplied by exposure and subtracted (x8192 + 8192)
 - gainframe-xN.pgm will be multiplied (1.0 = 16384)
 - clipframe-xN.pgm will be subtracted from highlights (x8)
 - reference images are 16-bit PGM, in the current directory
 - they are optional, but gain/clip frames require a dark frame
 - black ref columns will also be subtracted if you use a dark frame.
 
Creating reference images:
 - dark frames: average as many as practical, for each gain setting,
   with exposures ranging from around 1ms to 50ms:
        raw2dng --calc-darkframe *-gainx1-*.raw12 
 - DCNU (dark current nonuniformity) frames: similar to dark frames,
   just take a lot more images to get a good fit (use 256 as a starting point):
        raw2dng --calc-dcnuframe *-gainx1-*.raw12 
   (note: the above will compute BOTH a dark frame and a dark current frame)
 - gain frames: average as many as practical, for each gain setting,
   with a normally exposed blank OOF wall as target, or without lens
   (currently used for pattern noise reduction only):
        raw2dng --calc-gainframe *-gainx1-*.raw12 
 - clip frames: average as many as practical, for each gain setting,
   with a REALLY overexposed blank out-of-focus wall as target:
        raw2dng --calc-clipframe *-gainx1-*.raw12 
 - Always compute these frames in the order listed here
   (dark/dcnu frames, then gain frames (optional), then clip frames (optional).
 
General options:
--black=%d          : Set black level (default: 128)
                      - negative values allowed
--white=%d          : Set white level (default: 4095)
                      - if too high, you may get pink highlights
                      - if too low, useful highlights may clip to white
--width=%d          : Set image width (default: 4096)
--height=%d         : Set image height
                      - default: autodetect from file size
                      - if input is stdin, default is 3072
--swap-lines        : Swap lines in the raw data
                      - workaround for an old Beta bug
--hdmi              : Assume the input is a memory dump
                      used for HDMI recording experiments
--pgm               : Expect 16-bit PGM input from stdin
 
--lut               : Use a 1D LUT (lut-xN.spi1d, N=gain, OCIO-like)
 
--totally-raw       : Copy the raw data without any manipulation
                      - metadata and pixel reordering are allowed.
 
Pattern noise correction:
--rnfilter=1        : FIR filter for row noise correction from black columns
--rnfilter=2        : FIR filter for row noise correction from black columns
                      and per-row median differences in green channels
--fixrn             : Fix row noise by image filtering (slow, guesswork)
--fixpn             : Fix row and column noise (SLOW, guesswork)
--fixrnt            : Temporal row noise fix (use with static backgrounds; recommended)
--fixpnt            : Temporal row/column noise fix (use with static backgrounds)
--no-blackcol-rn    : Disable row noise correction from black columns
                      (they are still used to correct static offsets)
--no-blackcol-ff    : Disable fixed frequency correction in black columns
 
Flat field correction:
--dchp              : Measure hot pixels to scale dark current frame
--no-darkframe      : Disable dark frame (if darkframe-xN.pgm is present)
--no-dcnuframe      : Disable dark current frame (if dcnuframe-xN.pgm is present)
--no-gainframe      : Disable gain frame (if gainframe-xN.pgm is present)
--no-clipframe      : Disable clip frame (if clipframe-xN.pgm is present)
--no-blackcol       : Disable black reference column subtraction
                      - enabled by default if a dark frame is used
                      - reduces row noise and black level variations
--calc-darkframe    : Average a dark frame from all input files
--calc-dcnuframe    : Fit a dark frame (constant offset) and a dark current frame
                      (exposure-dependent offset) from files with different exposures
                      (starting point: 256 frames with exposures from 1 to 50 ms)
--calc-gainframe    : Average a gain frame (aka flat field frame)
--calc-clipframe    : Average a clip (overexposed) frame
--check-darkframe   : Check image quality indicators on a dark frame
 
Debug options:
--dump-regs         : Dump sensor registers from metadata block (no output DNG)
--fixpn-dbg-denoised: Pattern noise: show denoised image
--fixpn-dbg-noise   : Pattern noise: show noise image (original - denoised)
--fixpn-dbg-mask    : Pattern noise: show masked areas (edges and highlights)
--fixpn-dbg-col     : Pattern noise: debug columns (default: rows)
--export-rownoise   : Export row noise data to octave (rownoise\_data.m)
--get-pixel:%d,%d   : Extract one pixel from all input files, at given coordinates,
                      and save it to pixel.csv, including metadata. Skips DNG output.
\end{lstlisting}
                       
                      
Example:                       

\consoleCommand{./raw2dng --fixrnt --pgm --black=120 frame%05d.dng}

\textbf{Compiling raw2dng}\\

Compiling raw2dng on a 64bit system requires the gcc-multilib package.\\

\textbf{Ubuntu:} 

    \consoleCommand{sudo apt-get install gcc-multilib}   
    
    
\textbf{openSUSE:}

    \consoleCommand{sudo zypper install gcc-32bit libgomp1-32bit}  
    
    
\textbf{AXIOM Beta:}

1. Acquire the source from \href{https://github.com/apertus-open-source-cinema/misc-tools-utilities/tree/master/raw2dng}{https://github.com/apertus-open-source-cinema/misc-tools-utilities/tree/master/raw2dng}.\\
2. Copy files to AXIOM Beta.\\
3. Remove \importantKeyword{-m32} from Makefile.\\
4. Run \importantKeyword{make} inside camera.\\                     
                      
