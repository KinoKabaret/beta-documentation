\section{Associated Use-cases}

Overview text required.




\subsection{Configuration for Photography}

To increase the quality of created images it is highly recommended to do Factory Calibration of your AXIOM Beta first. See 6.3.1.2.\\




\subsubsection{Calibration}

\textbf{Compile raw2dng inside AXIOM Beta}\\

1. Acquire the source files from \href{https://github.com/apertus-open-source-cinema/misc-tools-utilities/tree/master/raw2dng}{https://github.com/apertus-open-source-cinema/misc-tools-utilities/tree/master/raw2dng}.\\
2. Copy files to AXIOM Beta.\\
3. Run \importantKeyword{make} inside camera. \\

\textbf{Install dcraw inside AXIOM Beta}\\

\consoleCommand{    pacman -Sy
    pacman -S dcraw}
    
\textbf{Picture Snapping script}\\

Download \href{https://github.com/apertus-open-source-cinema/beta-software/blob/master/beta-scripts/picture_snap.sh}{https://github.com/apertus-open-source-cinema/beta-software/blob/master/beta-scripts/picture\_snap.sh}\\ 

... to camera.\\

Make it executable:      

\consoleCommand{chmod +x picture\_snap.sh}



\subsubsection{Operation}


Picture snaps are saved in subfolders in the cameras \importantKeyword{/opt/picture-snap/} directory.\\

The script automatically names each subfolder and image by a timestamp: \importantKeyword{\%Y\%m\%d\_\%H\%M\%S} .\\

Execute the image snapping by providing the exposure time running eg.: 

\consoleCommand{./picture\_snap.sh 20ms}

One complete capture process plus DNG \& JPG conversion takes 18 seconds currently.\\
