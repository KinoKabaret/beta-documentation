\section{Writing Images}

For writing uncompressed full resolution full bitdepth raw images the AXIOM Beta uses a software called cmv\_snap3.\\

It is located in the /root/ directory and writes the images data directly to STDOUT.\\

cmv\_snap3 writes images in the \href{https://wiki.apertus.org/index.php/RAW12}{RAW12} format. Writing one image takes a few seconds depending on where the image is written to so this method is not viable for recording video footage other than timelapse. 




\subsection{Capture Still Images}
\subsubsection{Parameters}

The following parameters are available:

\begin{lstlisting}[language=bash,morekeywords=$,keywordstyle=\bfseries,frame=none,xleftmargin=.25in,belowskip=2em, aboveskip=2em]
    ./cmv_snap3 -h
    This is ./cmv_snap3 V1.10
    options are:
    -h        print this help message
    -8        output 8 bit per pixel
    -2        output 12 bit per pixel
    -d        dump buffer memory
    -b        enable black columns
    -p        prime buffer memory
    -r        dump sensor registers
    -t        enable cmv test pattern
    -z        produce no data output
    -e <exp>  exposure times
    -v <exp>  exposure voltages
    -s <num>  shift values by <num>
    -S <val>  writer byte strobe
    -R <fil>  load sensor registers
\end{lstlisting} 

\textbf{Examples}\\

Images can be written directly to the cameras internal micro SD card like this (10 milliseconds exposure time, 16bit):

\begin{lstlisting}[language=bash,morekeywords=$,keywordstyle=\bfseries,frame=none,xleftmargin=.25in,belowskip=2em, aboveskip=2em]
./cmv_snap3 -e 10ms > image.raw16
\end{lstlisting} 

Write image plus metadata (sensor configuration) to cameras internal micro SD card (20 milliseconds exposure time, 12bit): 

\begin{lstlisting}[language=bash,morekeywords=$,keywordstyle=\bfseries,frame=none,xleftmargin=.25in,belowskip=2em, aboveskip=2em]
./cmv_snap3 -2 -r -e 20ms > image.raw12
\end{lstlisting} 

You can also use cmv\_snap3 to change to exposure time (to 5 milliseconds in this example) without actually capturing an image, for the that -z parameter is used to not produce any data output: 

\begin{lstlisting}[language=bash,morekeywords=$,keywordstyle=\bfseries,frame=none,xleftmargin=.25in,belowskip=2em, aboveskip=2em]
./cmv_snap3 -z -e 5ms
\end{lstlisting} 

That cmv\_snap3 writes data to STDOUT makes it very versatile, we can for example capture images from and to a remote Linux machine connected to the Beta via Ethernet easily (lets assume the AXIOM Betas camera IP is set up as: 192.168.0.9 - SSH access has to be set up for this to work with a keypair) 

\begin{lstlisting}[language=bash,morekeywords=$,keywordstyle=\bfseries,frame=none,xleftmargin=.25in,belowskip=2em, aboveskip=2em]
ssh root@192.168.0.9 "./cmv_snap3 -2 -r -e 10ms" > snap.raw12
\end{lstlisting} 

To pipe the data into a file and display it at the same time with imagemagick on a remote machine: 

\begin{lstlisting}[language=bash,morekeywords=$,keywordstyle=\bfseries,frame=none,xleftmargin=.25in,belowskip=2em, aboveskip=2em]
    ssh root@192.168.0.9 "./cmv_snap3 -2 -r -e 10ms" | tee snap.raw12 | display -size 4096x3072 -depth 12 gray:-
\end{lstlisting} 

Use imagemagick to convert raw12 file into a color preview image: 

\begin{lstlisting}[language=bash,morekeywords=$,keywordstyle=\bfseries,frame=none,xleftmargin=.25in,belowskip=2em, aboveskip=2em]
cat test.raw12 | convert \( -size 4096x3072 -depth 12 gray:- \) \( -clone 0 -crop -1-1 \) \( -clone 0 -crop -1+0 \) \( -clone 0 -crop +0-1 \) -sample 2048x1536 \( -clone 2,3 -average \) -delete 2,3 -swap 0,1 +swap -combine test_color.png
\end{lstlisting} 


With raw2dng compiled inside the camera you can capture images directly to DNG, without saving the raw12: 


\begin{lstlisting}[language=bash,morekeywords=$,keywordstyle=\bfseries,frame=none,xleftmargin=.25in,belowskip=2em, aboveskip=2em]
./cmv_snap3 -2 -b -r -e 10ms | raw2dng snap.DNG
\end{lstlisting} 


\textbf{Note:} Supplying exposure time as parameter is required otherwise cmv\_snap3 will not capture an image. The exposure time can be supplied in "s" (seconds), "ms" (milliseconds), "us" (microseconds) and "ns" (nanoseconds). Decimal values also work (eg. "15.5ms"). 


\subsection{Image Overlays}

This section covers the mimg version 1.8 (see \href{https://github.com/apertus-open-source-cinema/beta-software/tree/master/mimg}{GitHub repo}), not the previous 1.6. 

AXIOM Beta features a full HD framebuffer that can be altered from the Linux userspace and is automatically "mixed" with the real time video from the image sensor on HDMI outputs.\\

The overlay could also be used to draw live histograms/scopes/HUD or menus.\\

The overlay in more recent firmware revisions supports alpha channel transparency. \\


\subsubsection{Internals}


The raw memory image is saved the following way: <ch0/12bit><ch1/12bit><ch2/12bit><ch3/12bit><overlay/16bit> \\

This means that for every 4 sensels there is only one overlay pixel. Meaning, while the CMV12000 image sensor is 4Kx3K the overlay is Full HD (1080p). 


\subsubsection{Prepare Images}

Convert 24bit PNG (1920x1080 pixels) with transparency to AXIOM Beta raw format: 

\begin{lstlisting}[language=bash,morekeywords=$,keywordstyle=\bfseries,frame=none,xleftmargin=.25in,belowskip=2em, aboveskip=2em]
convert input.png rgba:output.rgb
\end{lstlisting}

\textbf{Note :} This image format is different from \href{https://wiki.apertus.org/index.php/RAW12}{RAW12}.


\subsubsection{mimg}

mimg is software running on the camera that's used to load/alter anything related to overlays or test images.\\

Source code is available on \href{https://github.com/apertus-open-source-cinema/beta-software/tree/master/mimg}{GitHub}.

\begin{lstlisting}[language=bash,morekeywords=$,keywordstyle=\bfseries,frame=none,xleftmargin=.25in,belowskip=2em, aboveskip=2em]
    This is ./mimg V1.8
    options are:
    -h        print this help message
    -a        load all buffers
    -o        load as overlay
    -O        load as color overlay
    -r        load raw data
    -w        use word sized data
    -D <val>  image color depth
    -W <val>  image width
    -H <val>  image height
    -P <val>  uniform pixel color
    -T <val>  load test pattern
    -B <val>  memory mapping base
    -S <val>  memory mapping size
    -A <val>  memory mapping address
\end{lstlisting} 

\textbf{Examples:}\\


Clear overlay: 

\begin{lstlisting}[language=bash,morekeywords=$,keywordstyle=\bfseries,frame=none,xleftmargin=.25in,belowskip=2em, aboveskip=2em]
./mimg -a -o -P 0
\end{lstlisting} 


Load monochrome overlay: 

\begin{lstlisting}[language=bash,morekeywords=$,keywordstyle=\bfseries,frame=none,xleftmargin=.25in,belowskip=2em, aboveskip=2em]
./mimg -o -a file.raw
\end{lstlisting} 


Load color overlay: 

\begin{lstlisting}[language=bash,morekeywords=$,keywordstyle=\bfseries,frame=none,xleftmargin=.25in,belowskip=2em, aboveskip=2em]
./mimg -O -a file.raw
\end{lstlisting} 


Enable overlay: 

\begin{lstlisting}[language=bash,morekeywords=$,keywordstyle=\bfseries,frame=none,xleftmargin=.25in,belowskip=2em, aboveskip=2em]
gen_reg 11 0x0104F000
\end{lstlisting} 


Disable overlay: 

\begin{lstlisting}[language=bash,morekeywords=$,keywordstyle=\bfseries,frame=none,xleftmargin=.25in,belowskip=2em, aboveskip=2em]
gen_reg 11 0x0004F000
\end{lstlisting} 


\textbf{Old overlays}:\\

Overlays for mimg 1.6 are not compatible anymore. You can use

\begin{lstlisting}[language=bash,morekeywords=$,keywordstyle=\bfseries,frame=none,xleftmargin=.25in,belowskip=2em, aboveskip=2em]
convert -size 1920x1080 -depth 6 rgba:overlay_04.rgb overlay_04.png
\end{lstlisting} 

... to convert them to PNG before continuing with the preparation above. 